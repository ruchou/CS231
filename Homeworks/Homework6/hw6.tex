\documentclass{article}
\topmargin = -0.5in
\oddsidemargin = 0in
\evensidemargin = \oddsidemargin
\textwidth = 6.5in
\textheight = 9in
\usepackage{times}
\usepackage{bcprules}
\usepackage{amsthm}
\usepackage{amsmath}
\usepackage{listings}
\usepackage{mathtools}
\usepackage{amssymb}
\usepackage[T1]{fontenc}
\usepackage{textcomp}
\usepackage{xcolor}

\lstset{
    language=caml,
    columns=[c]fixed,
% basicstyle=\small\ttfamily,
    keywordstyle=\bfseries,
    upquote=true,
    commentstyle=,
    breaklines=true,
    showstringspaces=false,
    stringstyle=\color{blue},
    literate={'"'}{\textquotesingle "\textquotesingle}3
}

\newcommand{\mathType}[2]{\mathtt{#1}:\mathtt{#2}}

\newcommand{\step}[2]{{\tt #1} $\longrightarrow$ {\tt #2}}

\newcommand{\inferrule}[3]{\infrule[#1]{\mbox{#2}}{\mbox{#3}}}
\newcommand{\inferax}[2]{\infrule[#1]{\mbox{}}{\mbox{#2}}}

\newtheorem{example}{Example}

\title{Homework 6}

\author{Yan-Ru Jhou, Nurrachman Liu}

\date{Mar 3rd, 2020}

\begin{document}

    \maketitle

    \begin{description}
        \item[1]{Sol}\\
            \begin{enumerate}
                \item[(a)]{Ans}\\
                \begin{lstlisting}
let r = ref 41 in
let x = (r := !r+1)in
    !r ;;
                \end{lstlisting}
                \item[(b)]{Ans} \\
                \begin{lstlisting}
let r = ref 41 in
let x = ((function r:Ref Int -> (r := 41; 500)) (let _ = (r := !r+1) in ref !r)) in
    !r
                \end{lstlisting}
                \item[(c)]{Ans} \\
                \begin{lstlisting}
let f=(let x = ref (-40) in
        function () -> (x:=!x+41;!x)) in
                (f ()) * (f ())
                \end{lstlisting}
            \end{enumerate}
        \item [2]{Sol}
            \begin{enumerate}
                \item[(a)]{No}\\
                $\text{let } a = \text{ref } 10 \text{ in} (free\, a;!a)$ \\ This would typecheck to $Int$ but it would eventually stuck
                \item[(b)]{Yes}\\
                \item[(c)]{No} Consider the following example\\
                $\Sigma=\{(l,Int)\}$,\;  $\mu=\{(l,12)\}$\;
                $t=(free\; l;!l)$ then it would step to\\
                $t'=(!l)$, where $u'=\{\},\Sigma'=\{\}$.
                However, it would not typecheck
            \end{enumerate}
        \item[3]{Sol}
            \begin{enumerate}
                \item [(a)]{Yes}
                \item[(b)]{No}\\
                $\ (function \text{ x}: \text{Ref }Top \rightarrow !x)(y:Top)$
                \item[(c)] {No}\\
                $\text{let } x: \text{Ref Top} = \text{Ref } 1 \, in $\\
                $\text{let } y: \text{Ref (Top,Top)} = x\, in$\\
                $fst\, !y$
                \item[(d)] {No}\\
                $\text{let } x: \text{Ref (Top,Top)} = \text{Ref } (1,2) \, in $\\
                $\text{let } y: \text{Ref Top} = x \,in$\\
                $y:=false;$\\
                $fst\, !x$
            \end{enumerate}
    \end{description}

%    \section{Language of Booleans and Integers}
%
%    \subsection{Syntax}
%
%    \begin{tt}
%        \begin{tabular}{rrl}
%            t & ::= & b $\mid$ if t then t else t \\
%            &$\mid$& n $\mid$ t + t $\mid$ t > t \\
%            b & ::= & true $\mid$ false \\
%            n & ::= & \mbox{\rm integer constant} \\
%            v & ::= & b $\mid$ n \\
%        \end{tabular}
%    \end{tt}
%
%
%    \subsection{Small-Step Operational Semantics}
%
%
%    \inferax{E-IfTrue}{\step{if true then t$_2$ else t$_3$}{t$_2$}}
%
%    \inferax{E-IfFalse}{\step{if false then t$_2$ else t$_3$}{t$_3$}}
%
%    \inferrule{E-If}{\step{t$_1$}{t$_1'$}}
%    {\step{if t$_1$ then t$_2$ else t$_3$}{if t$_1'$ then t$_2$ else t$_3$}}
%
%
%    \inferrule{E-Plus1}{\step{t$_1$}{t$_1'$}}
%    {\step{t$_1$ + t$_2$}{t$_1'$ + t$_2$}}
%
%
%    \inferrule{E-Plus2}{\step{t$_2$}{t$_2'$}}
%    {\step{v$_1$ + t$_2$}{v$_1$ + t$_2'$}}
%
%
%    \inferrule{E-PlusRed}{n = n$_1$ $[\![+]\!]$ n$_2$}
    %%    {\step{n$_1$ + n$_2$}{n}}
%
%    \inferrule{E-GT1}{\step{t$_1$}{t$_1'$}}
%    {\step{t$_1$ > t$_2$}{t$_1'$ > t$_2$}}
%
%
%    \inferrule{E-GT2}{\step{t$_2$}{t$_2'$}}
%    {\step{v$_1$ > t$_2$}{v$_1$ > t$_2'$}}
%
%
%    \inferrule{E-GTRed}{b = n$_1$ $[\![>]\!]$ n$_2$}
%    {\step{n$_1$ > n$_2$}{b}}


\end{document}


%F = X0,X = X1,F = X → X2,F = X2 → X4,X5 = X0 → X4,X6 = X0 → X5
%
%(F=X0, X = X1; F= X -> X2; F = X2 -> X4 ; X5 = X0 -> X4; X6 = X0 -> X5  )