\documentclass{article}
\topmargin = -0.5in
\oddsidemargin = 0in
\evensidemargin = \oddsidemargin
\textwidth = 6.5in
\textheight = 9in
\usepackage{times}
\usepackage{bcprules}
\usepackage{amsthm}
\usepackage{amsmath}
\newcommand{\mathType}[2]{\mathtt{#1}:\mathtt{#2}}

\newcommand{\step}[2]{{\tt #1} $\longrightarrow$ {\tt #2}}

\newcommand{\inferrule}[3]{\infrule[#1]{\mbox{#2}}{\mbox{#3}}}
\newcommand{\inferax}[2]{\infrule[#1]{\mbox{}}{\mbox{#2}}}

\newtheorem{example}{Example}

\title{Homework 4}

\author{Yan-Ru Jhou, Nurrachman Liu}

\date{Feb 14th, 2020}

\begin{document}

    \maketitle

    \begin{description}
        \item[1]{Sol}\\
            \begin{enumerate}
                \item[(a)] { $((A \wedge B) \wedge C) \rightarrow ( A \wedge ( B \wedge C) )$}\\
                    $ \text{function}\, p:((A \wedge B) \wedge C) \rightarrow ( (fst\,(fst\,p))\; \wedge\; (snd\,(fst\,p) \wedge snd\,c) ) $
                \item[(b)] {$((A \vee B)\; \vee \; (C))  \rightarrow ( A \; \vee \; (B \vee C))$ }\\
                    $\text{function}\, p:(A \vee B)\; \vee \; (C))  \rightarrow ((left \: (left\, p)) \; \vee \; ( (right \: (left \: p))  \: \vee \: ( right p) ) )  $
                \item[(c)]{$((A \; \vee B) \longrightarrow C)$}\\
                    $\text{function}\, p:((A \; \vee B) \rightarrow C) \longrightarrow (\text{function a:A} \rightarrow p \, a \; \wedge\; \text{function b:B}\rightarrow p \, b )$
                \item[(c)] {$(A \rightarrow B) \longrightarrow (\neg A \rightarrow \neg B)$}
                    $\text{function f:}(B \rightarrow false) \longrightarrow \text{function a:A} \longrightarrow f\: (p\: a)$
            \end{enumerate}

        \item [5]{Sol}
            \begin{enumerate}
                \item[(a)]{Proof of Progressive Theorem} \\
                    By induction on the derivation of $\emptyset \vdash t:T$\\
                    \textit{Induction hypothesis:} If $\emptyset \vdash t_0:T_0$, where $\emptyset \vdash t_0:T_0$ is a sub-derivation of $\emptyset \vdash t:T$, then either $t_0$ is a value or exists $t_0'$ such that  $t_0$ $\rightarrow$ $t_0'$
                    \begin{proof}
                        Case analysis on the root rule in $\emptyset \vdash t:T$.
                        \begin{itemize}
                            \item Case T-VAR: Then t = x, which is a value
                            \item Case T-FUN: Then t = function x:T $\rightarrow$ t , which is a value
                            \item Case T-APP: Then t = $t_1$ $t_2$, where $\emptyset \vdash t_1 t_2 :T$, $\emptyset \vdash t_1:T2 \rightarrow T $, and $\emptyset \vdash t_2:T2$. \\
                                By IH, $t_1$ is either a value or there exists $t_1'$ such that $t_1$ $\rightarrow$ $t_1'$, and $t_2$ is either a value or there exists $t_2'$ such that $t_2$ $\rightarrow$ $t_2'$\\
                                IF $t_1$ is a value, by CFL, $t_1$ has the value $v_1$ with the form $function\ x:T_{12} \rightarrow t_{12}$\\
                                IF $t_2$ is a value, by CFL, $t_2$ has the value $v_2$ with the form $function\ x:T_{22} \rightarrow t_{22}$\\
                                Case analysis on $t_1$ and $t_2$
                                \begin{itemize}
                                    \item Case exists $t_1'$ such that $t_1$ $\rightarrow$ $t_1'$:\\
                                        Then $t_1\ t_2$ $\rightarrow$ $t_1'\ t_2$ by E-APP1
                                    \item Case $t_1$ is a value $v_1$ and exists $t_2'$ such that $t_2$ $\rightarrow$ $t_2'$\\
                                        Then $t \rightarrow t' = v_1 t_2'$ by E-APP2
                                    \item Case Both $t_1$ and $t_2$ are values\\
                                        Then, $t \rightarrow t' = [x \rightarrow v_2]t_{12}$ by E-APPBETA
                                \end{itemize}
                        \end{itemize}
                    \end{proof}
                \item[(b)]{Arbitrary Type Environment}\\No\\ let $t_1 = x$ , $t_2 = \text{function \ } y:Bool \rightarrow  y $, and $\Gamma = \{(x,(Bool \rightarrow Bool) \rightarrow Bool)\}$\\
                   Then $\Gamma \vdash t_1 t_2: bool$ , but $t_1 t_2$ cannot step and it is not a value
                \item[(c)]{Proof of Preservation Theorem}\\
                    By induction on the derivation of $\emptyset \vdash t:T$\\
                    \textit{Induction hypothesis:} If $\emptyset \vdash t_0:T_0$ and $t_0$ $\rightarrow$ $t_0'$, where $\emptyset \vdash t_0:T_0$ is a sub-derivation of $\emptyset \vdash t:T$, then $\emptyset \vdash t_0':T_0$
                    \begin{proof}
                        Case analysis on the root rule in $\emptyset \vdash t:T$.
                        \begin{itemize}
                            \item Case T-VAR: Then $t = x$,and $x:T \in \emptyset$. Contradiction ($t$ cannot step)
                            \item Case T-FUN: Then $t = \text{function\ } x:T_1 \rightarrow t_2 $.Contradiction($t$ cannot step)
                            \item Case T-APP: Then $t = t_1 t_2$, $\emptyset \vdash t_1: T_{11} \rightarrow T_{12}$,\space  $\emptyset \vdash t_2:T_{11}$,\space and $T=T_{12}$\\
                            Subcase analysis on $t \rightarrow t'$
                            \begin{itemize}
                                \item Case E-APP1: Then $t_1 \rightarrow t_1'$ , $t' = t_1't_2$\\
                                    Since $\emptyset \vdash t_1:T_{11} \rightarrow T_{12}$,and $t_1 \rightarrow t_1'$,\\
                                    By IH, $\emptyset \vdash t1':T_{11}\rightarrow T_{12}$.
                                    So, $\emptyset \vdash t':T_{12} = T$ by T-APP
                                \item Case E-APP2: Then $t_1 = v_1$, $t_2 \rightarrow t_2'$, $t'=v_1 v_2'$\\ Similar
                                \item Case E-APPBETA: Then $t_1 = \text{function\ } x:T_{11} \rightarrow t_{12}$, $t_2 = v_{12}$, $t' = [x \rightarrow v_{12}]t_{12}$\\
                                    By Inversion Lemma, $\emptyset, x:T_{12} \vdash t_{12}:T_{12}$.\\So, by type substitution lemma, $\emptyset \vdash [x \rightarrow v_{12}]:T_{12} = T$
                            \end{itemize}
                        \end{itemize}
                    \end{proof}
            \end{enumerate}
        \item[6]{Sol}
            \begin{enumerate}
                \item [(a)]{Small-Step Operational Semantics }
                    \inferrule{E-FOR1}{\step{t$_1$}{t$_1'$}}
                    {\step{for t$_1$ do t$_2$}{for t$_1'$ do t$_2$}}
                    \inferrule{E-FORRED1}{ cond = n $[\![>]\!]$ 0 \andalso n' = n $[\![-]\!]$ 1 }
                    {\step{for n do t$_2$}{ if cond then for n' do t$_2$ else ()}}
                \item[(b)]{Typing Rule }
                    \inferrule{T-FOR}{$\Gamma \vdash t_1:\text{int}$ \andalso $\Gamma \vdash t_2:\text{T}$}
                    { $\Gamma \vdash$ for t$_1$ do t$_2$:Unit}
            \end{enumerate}

    \end{description}

%    \section{Language of Booleans and Integers}
%
%    \subsection{Syntax}
%
%    \begin{tt}
%        \begin{tabular}{rrl}
%            t & ::= & b $\mid$ if t then t else t \\
%            &$\mid$& n $\mid$ t + t $\mid$ t > t \\
%            b & ::= & true $\mid$ false \\
%            n & ::= & \mbox{\rm integer constant} \\
%            v & ::= & b $\mid$ n \\
%        \end{tabular}
%    \end{tt}
%
%
%    \subsection{Small-Step Operational Semantics}
%
%
%    \inferax{E-IfTrue}{\step{if true then t$_2$ else t$_3$}{t$_2$}}
%
%    \inferax{E-IfFalse}{\step{if false then t$_2$ else t$_3$}{t$_3$}}
%
%    \inferrule{E-If}{\step{t$_1$}{t$_1'$}}
%    {\step{if t$_1$ then t$_2$ else t$_3$}{if t$_1'$ then t$_2$ else t$_3$}}
%
%
%    \inferrule{E-Plus1}{\step{t$_1$}{t$_1'$}}
%    {\step{t$_1$ + t$_2$}{t$_1'$ + t$_2$}}
%
%
%    \inferrule{E-Plus2}{\step{t$_2$}{t$_2'$}}
%    {\step{v$_1$ + t$_2$}{v$_1$ + t$_2'$}}
%
%
%    \inferrule{E-PlusRed}{n = n$_1$ $[\![+]\!]$ n$_2$}
    %%    {\step{n$_1$ + n$_2$}{n}}
%
%    \inferrule{E-GT1}{\step{t$_1$}{t$_1'$}}
%    {\step{t$_1$ > t$_2$}{t$_1'$ > t$_2$}}
%
%
%    \inferrule{E-GT2}{\step{t$_2$}{t$_2'$}}
%    {\step{v$_1$ > t$_2$}{v$_1$ > t$_2'$}}
%
%
%    \inferrule{E-GTRed}{b = n$_1$ $[\![>]\!]$ n$_2$}
%    {\step{n$_1$ > n$_2$}{b}}


\end{document}
