\documentclass{article}
\topmargin = -0.5in
\oddsidemargin = 0in
\evensidemargin = \oddsidemargin
\textwidth = 6.5in
\textheight = 9in
\usepackage{times}
\usepackage{bcprules}
\usepackage{amsthm}
\usepackage{amsmath}
\usepackage{listings}
\usepackage{mathtools}
\usepackage{amssymb}

\newcommand{\mathType}[2]{\mathtt{#1}:\mathtt{#2}}

\newcommand{\step}[2]{{\tt #1} $\longrightarrow$ {\tt #2}}

\newcommand{\inferrule}[3]{\infrule[#1]{\mbox{#2}}{\mbox{#3}}}
\newcommand{\inferax}[2]{\infrule[#1]{\mbox{}}{\mbox{#2}}}

\newtheorem{example}{Example}

\title{Homework 4}

\author{Yan-Ru Jhou, Nurrachman Liu}

\date{Feb 14th, 2020}

\begin{document}

    \maketitle

    \begin{description}
        \item[1]{Sol}\\
            \begin{enumerate}
                \item[(a)] { $((A \wedge B) \wedge C) \rightarrow ( A \wedge ( B \wedge C) )$}\\
                    $ \text{function}\, p:((A \wedge B) \wedge C) \rightarrow ( (fst\,(fst\,p))\; , \; (snd\,(fst\,p) , snd\, p) ) $
                \item[(b)] {$((A \vee B)\; \vee \; (C))  \rightarrow ( A \; \vee \; (B \vee C))$ }\\
                    $\text{function}\, p:(A \vee B)\; \vee \; (C))  \longrightarrow  \text{match } \text{Left } p \text{ with} \\ \text{Left } a \rightarrow \\ | \text{Right } b \rightarrow \text{match } p \text{ with} \\ ~~~~~\text{Right } c \rightarrow c \\  ~~~~~|\text{\_} \rightarrow b $
                \item[(c)]{$((A \; \vee B) \longrightarrow C)$}\\
                    $\text{function}\, p:((A \; \vee B) \rightarrow C) \longrightarrow (\text{function a:A} \rightarrow p \, a \; \wedge\; \text{function b:B}\rightarrow p \, b )$
                \item[(c)] {$(A \rightarrow B) \longrightarrow (\neg A \rightarrow \neg B)$}
                    $\text{function f:}(B \rightarrow false) \longrightarrow \text{function a:A} \longrightarrow f\: (p\: a)$
            \end{enumerate}
        \item [2]{Sol}
            \begin{enumerate}
                \item[(a)]{Small-Step} \\
                    \inferax{E-While}{\step{while t${_1}$ do t${_2}$}{if t${_1}$ then t${_2}$;while t${_1}$ do t${_2}$ else t${_2}$}}
                \item[(b)]{Typing Rule} \\
                \inferrule{T-While}{$\Gamma \vdash t_1 : bool$ \andalso $\Gamma \vdash t_2:T$ }
                {while t${_1}$ do t${_2}$: True}
            \end{enumerate}
        \item[3]{Sol}
            \begin{enumerate}
                \item [(a)]{WhileFun}\\
                    $\text{letrec}\: whileFun = \\ \text{function }t{_1}:(True \rightarrow Bool) \longrightarrow \text{function }t{_2}:(True -> True) \\ in \, \text{if } t_1 \; \text{then } t_2; whileFun \;\; t_1\,t_2 \text{else} ()$
            \end{enumerate}
        \item[(4)]{Sol}
            \begin{enumerate}
                \item[(a)]{Small-Step}\\
                    \inferrule{E-Node1}{\step{t${_1}$}{t$_1'$}}
                    {\step{node(t$_1$,t${_2}$,t${_3}$) }{node(t${_1'}$,t${_2}$,t${_3}$)}}
                    \inferrule{E-Node2}{\step{t${_2}$}{t$_2'$}}
                    {\step{node(tv$_1$,t${_2}$,t${_3}$) }{node(tv${_1}$,t${_2'}$,t${_3}$)}}
                    \inferrule{E-Node3}{\step{t${_3}$}{t$_3'$}}
                    {\step{node(tv$_1$,t${_2}$,t${_3}$) }{node(tv${_1}$,t${_2}$,t${_3'}$)}}
                    \inferrule{E-LeftTree}{\step{t$$}{t$'$}}
                    {\step{leftTree t }{leftTree t${'}$}}
                    \inferax{E-LeftTreeRed}{\step{leftTree Node(tv${_1}$,v${_2}$,tv${_3}$)}{tv${_1}$}}
                    \inferrule{E-RightTree}{\step{t$$}{t$'$}}
                    {\step{RightTree t }{RightTree t${'}$}}
                    \inferax{E-RightTreeRed}{\step{rightTree Node(tv${_1}$,v${_2}$,tv${_3}$)}{tv${_3}$}}
                    \inferrule{E-data1}{\step{t$$}{t$'$}}
                    {\step{data t }{data t${'}$}}
                    \inferax{E-dataRed}{\step{data node(tv${_1}$,t${_2}$,tv${_3}$)}{t${_2}$}}
                \item[(b)]{Typing Rules}\\
                    \inferax{T-leaf}{$\Gamma \vdash \text{leaf}:False \, \text{Tree}$}
                    \inferrule{T-Node}{$\Gamma \vdash t_1: \text{T Tree}$ \andalso $\Gamma \vdash t_2: T$ \andalso $\Gamma \vdash t_3:\text{T Tree}$ }{$\Gamma \vdash\ node(t_1,t_2,t_3): \text{T Tree} $}
                    \inferrule{T-leftTree}{$\Gamma \vdash t: \text{T Tree}$}{$\Gamma \vdash \text{leftTree}\, t: \text{T Tree}$}
                    \inferrule{T-rightTree}{$\Gamma \vdash t: \text{T Tree}$}{$\Gamma \vdash \text{rightTree}\, t: \text{T Tree}$}
                    \inferrule{T-Data}{ $\Gamma \vdash$ v${_2}$:T}{$\Gamma \vdash$ node(tv${_1}$,v${_2}$,tv${_3}$): T Tree}
                \item[(c)]{Sol}\\
                    The typechecker needs to guess types since we allow leaf to be have type false.\\
                    We could define the tree type as Tree = leaf \textbar node(t1,t2,t3).
                    For example, in ocaml
                    \begin{lstlisting}[language=Caml]
type 'a tree =
| Leaf
| Node of 'a * 'a tree * 'a tree
                    \end{lstlisting}
                \item[(d)]{No}\\ It would violate Progress.
                if $data$ leaf then $0$ else $1$
            \end{enumerate}
        \item[(5)]{Sol}
            \begin{enumerate}
                \item [(a)]{Inference Rules}
                    \inferrule{I-Fun}
                    {$\frac{  \cfrac{ \cfrac{\text{fresh } F}{ \text{fresh } X{_3},\text{ tinfer}(\{ f=X{_0}, x=X{_1} \},f) = \{ F, \{ F=X{_0} \} \}} ,\qquad  \frac{\text{fresh} X{_2},  \text{tinfer}(\{ f:X{_0}, x:X{_1}\}, f) = ( F, \{ F=X{_0}\}), \frac{\text{fresh} X }{\text{infer}(\{ f:X{_0}, x:X{_1}  \}, x) = ( X, \{X=X{_1})}  }{  \text{tinfer}(\{ F=X{_0}, X=X{_1}\}, \text{f x}) = (X{_2}, \{F=X{_0}, X=X{_1}, F = X \rightarrow X{_2}\}) } }{ \text{fresh } X{_1},X{_4}, \text{ tinfer}(\{f=X{_0}, x=X{_1}\}, f (f x) ) = (X{_3},\{ F=X{_0}, X=X{_1}, F= X\rightarrow X{_2}, F =  X{_2} \rightarrow X{_3}\} )}}{ \text{fresh } X{_0},X{_5}, \text{ tinfer}(\{f:X{_0}\}, \text{function x} \rightarrow f (f x)) =  (X{_4},  \{ F=X{_0}, X=X{_1}, F= X\rightarrow X{_2}, F =  X{_2} \rightarrow X{_3},  X{_4}= X{_1} \rightarrow X{_3}  \} )}$}
                    {tinfer($\emptyset$, \text{function} f $\rightarrow$ \text{function} x $\rightarrow$ f(fx)) = ($X{_5}$, C )}
                    $ C = \{ F=X{_0}, X=X{_1}, F= X\rightarrow X{_2}, F =  X{_2} \rightarrow X{_3},  X{_4}= X{_1} \rightarrow X{_3}, X{_5}= X{_0} \rightarrow X{_4} \}$ \\
                \item[(b)]{Sol}\\
                    $[ F \mapsto(T \rightarrow T),X \mapsto T, X{_0} \mapsto T, X{_1} \mapsto T, X{_2} \mapsto T, X{_3} \mapsto T, X{_4} \mapsto(T \rightarrow T), X{_5} \mapsto (T \rightarrow T) \rightarrow (T \rightarrow T)]$\\
                    So, f could have the type $T \rightarrow T$, and the x could have the type $T$

            \end{enumerate}
        \item[(6)]{Sol} \\
            \inferrule{I-left}{tinfer($\Gamma$,s) = (S${_1}$,C${_1}$) \andalso fresh X \andalso fresh S${_r}$}
            {tinfer($\Gamma$,left s) = (X, C $\cup \{ S{_1} \vee S{_r}  \}$)}
            \inferrule{I-right}{tinfer($\Gamma$,s) = (S${_1}$, C${_1}$) \andalso fresh X \andalso fresh S${_l}$}
            {tinfer($\Gamma$,right s) = (X, C $\cup \{ S{_l} \vee S{_1} \}$)}
            \inferrule{I-match}
            {fresh x${_1}$, x${_2}$ \andalso tinfer($\Gamma$,s) = (S, C) \andalso tinfer($\{ \Gamma, x{_1}:X{_1} \} , s_{_1}$) = $(S{_1}, C{_1})$ \andalso tinfer($\{\Gamma, x{_2}:X{_2}\},s_{2}$) = ($S{_2},C{_2}$)}
            {tinfer($\Gamma$,match s with left s${_1}$ $\rightarrow$ s${_1}$ $|$ right x${_2}$ $\rightarrow$ s${_2}$)} = $(X, C \cup C{_2} \cup C{_3} \cup \{ S{_1} = S{_2}, S = S{_1} \vee S{_2} \} )$
    \end{description}

%    \section{Language of Booleans and Integers}
%
%    \subsection{Syntax}
%
%    \begin{tt}
%        \begin{tabular}{rrl}
%            t & ::= & b $\mid$ if t then t else t \\
%            &$\mid$& n $\mid$ t + t $\mid$ t > t \\
%            b & ::= & true $\mid$ false \\
%            n & ::= & \mbox{\rm integer constant} \\
%            v & ::= & b $\mid$ n \\
%        \end{tabular}
%    \end{tt}
%
%
%    \subsection{Small-Step Operational Semantics}
%
%
%    \inferax{E-IfTrue}{\step{if true then t$_2$ else t$_3$}{t$_2$}}
%
%    \inferax{E-IfFalse}{\step{if false then t$_2$ else t$_3$}{t$_3$}}
%
%    \inferrule{E-If}{\step{t$_1$}{t$_1'$}}
%    {\step{if t$_1$ then t$_2$ else t$_3$}{if t$_1'$ then t$_2$ else t$_3$}}
%
%
%    \inferrule{E-Plus1}{\step{t$_1$}{t$_1'$}}
%    {\step{t$_1$ + t$_2$}{t$_1'$ + t$_2$}}
%
%
%    \inferrule{E-Plus2}{\step{t$_2$}{t$_2'$}}
%    {\step{v$_1$ + t$_2$}{v$_1$ + t$_2'$}}
%
%
%    \inferrule{E-PlusRed}{n = n$_1$ $[\![+]\!]$ n$_2$}
    %%    {\step{n$_1$ + n$_2$}{n}}
%
%    \inferrule{E-GT1}{\step{t$_1$}{t$_1'$}}
%    {\step{t$_1$ > t$_2$}{t$_1'$ > t$_2$}}
%
%
%    \inferrule{E-GT2}{\step{t$_2$}{t$_2'$}}
%    {\step{v$_1$ > t$_2$}{v$_1$ > t$_2'$}}
%
%
%    \inferrule{E-GTRed}{b = n$_1$ $[\![>]\!]$ n$_2$}
%    {\step{n$_1$ > n$_2$}{b}}


\end{document}


%F = X0,X = X1,F = X → X2,F = X2 → X4,X5 = X0 → X4,X6 = X0 → X5
%
%(F=X0, X = X1; F= X -> X2; F = X2 -> X4 ; X5 = X0 -> X4; X6 = X0 -> X5  )