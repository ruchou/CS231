\documentclass{article}
\topmargin = -0.5in
\oddsidemargin = 0in
\evensidemargin = \oddsidemargin
\textwidth = 6.5in
\textheight = 9in
\usepackage{times}
\usepackage{bcprules}
\usepackage{amsthm}
\usepackage{amsmath}

\newcommand{\step}[2]{{\tt #1} $\longrightarrow$ {\tt #2}}

\newcommand{\inferrule}[3]{\infrule[#1]{\mbox{#2}}{\mbox{#3}}}
\newcommand{\inferax}[2]{\infrule[#1]{\mbox{}}{\mbox{#2}}}

\newtheorem{example}{Example}

\title{Homework 2}

\author{Yan-Ru Jhou, Nurrachman Liu}

\date{Jan 17th, 2020}

\begin{document}

    \maketitle

    \begin{description}
        \item[2]{Consider the small-step operational semantics for booleans and integers in Section1.2 of the cheatsheet.Provide a grammar for a new metavariable s that characterizes exactly the stuck expressions relative to this semantics the set of terms that are not values but cannot step. You can introduce other metavariables as needed.}\\
            \begin{tt}
                    \begin{tabular}{rrl}
                        t & ::= & b $\mid$ if t then t else t \\
                        &$\mid$& n $\mid$ t + t $\mid$ t > t \\
                        b & ::= & true $\mid$ false \\
                        n & ::= & \mbox{\rm integer constant} \\
                        v & ::= & b $\mid$ n \\
                        s & ::= & Stuck \\
                        nonBool & ::= & \mbox{\rm integer constant} \\
                        nonInt & ::= & true $\mid$ false \\
                    \end{tabular}
            \end{tt}
            \inferax{E-IfTrue}{\step{if true then t$_2$ else t$_3$}{t$_2$}}

            \inferax{E-IfFalse}{\step{if false then t$_2$ else t$_3$}{t$_3$}}

            \inferrule{E-If}{\step{t$_1$}{t$_1'$}}
            {\step{if t$_1$ then t$_2$ else t$_3$}{if t$_1'$ then t$_2$ else t$_3$}}

            \inferrule{E-IfSTuck}{\step{t$_1$}{NonBool}}
            {\step{if t$_1$ then t$_2$ else t$_3$}{Stuck}}

            \inferrule{E-Plus1}{\step{t$_1$}{t$_1'$}}
            {\step{t$_1$ + t$_2$}{t$_1'$ + t$_2$}}

            \inferrule{E-Plus1Stuck}{\step{t$_1$}{NonInt}}
            {\step{t$_1$ + t$_2$}{Stuck}}

            \inferrule{E-Plus2}{\step{t$_2$}{t$_2'$}}
            {\step{v$_1$ + t$_2$}{v$_1$ + t$_2'$}}

            \inferrule{E-Plus2Stuck}{\step{t$_2$}{NonInt}}
            {\step{v$_1$ + t$_2$}{Stuck}}

            \inferrule{E-PlusRed}{n = n$_1$ $[\![+]\!]$ n$_2$}
            {\step{n$_1$ + n$_2$}{n}}

            \inferrule{E-GT1}{\step{t$_1$}{t$_1'$}}
            {\step{t$_1$ > t$_2$}{t$_1'$ > t$_2$}}
            \inferrule{E-GT1}{\step{t$_1$}{NonInt}}
            {\step{t$_1$ > t$_2$}{Stuck}}

            \inferrule{E-GT2}{\step{t$_2$}{t$_2'$}}
            {\step{v$_1$ > t$_2$}{v$_1$ > t$_2'$}}
            \inferrule{E-GT2}{\step{t$_2$}{NonInt}}
            {\step{v$_1$ > t$_2$}{Stuck}}

            \inferrule{E-GTRed}{b = n$_1$ $[\![>]\!]$ n$_2$}
            {\step{n$_1$ > n$_2$}{b}}
        \item [3]{Sol}\\
            \begin{enumerate}
                \item [(a)]{New rules}
                    \inferrule{E-AND}{\step{t$_1$}{t$_1'$}}
                    {\step{t$_1$ \&\& t$_2$}{t$_1'$ \&\& t$_2$}}
                    \inferrule{E-ANDFalse}{v$_1$ $=$ false}
                    {\step{v$_1$ \&\& t$_2$}{false}}
                    \inferrule{E-ANDTrue}{v$_1$ $=$ true}
                    {\step{v$_1$ \&\& t$_2$}{t$_2$}}
                \item[(b)]{New rules for types}
                    \inferrule{T-AND}{\tt t$_1$:Bool \and t$_2$:Bool}
                    {\tt t$_1$ \&\& t$_2$:Bool}
                \item[(c)]{Proof of Progressive Theorem}
                \item[(d)]{Proof of Preservation Theorem}
            \end{enumerate}
        \item[4]{Sol}\\
            \begin{enumerate}
                \item [(a)]{New Operational Semantics}
                     \inferrule{E-AndToIF}{}
                    {\step{t$_1$ \&\& t$_2$}{if t$_1$ then t$_2$ else false}}
                \item[(b)]{}\\
                No, 
            \end{enumerate}
        \item[5] {Sol} \\
            \begin{enumerate}
                \item[(a)]{Remove the rule E-IFFALSE} \\
                    \begin{enumerate}
                        \item Progress: Consider the example: \verb|if false then true else true|\\It would violate Progress since it would try to step with E-If but it cannot take a step to t'
                        \item Preserve: It is still valid because E-IFFALSE would not take a step of evaluation.
                    \end{enumerate}
                \item[(b)]{Add the following axiom to the type system: $\frac{}{0:Bool}$}\\
                    \begin{enumerate}
                        \item Progress: Consider the example: \verb|if 0 then 231 else 232 |\\It would violate Progress since it would violate teh Canonical forms lemma which states that if a value of type Bool is either true of false. It would cause stuck in E-IFTrue or E-IFFalse.
                        \item Preserve: It is still valid because the term 0  would not take a step of evaluation.
                    \end{enumerate}
                \item[(c)]{ Add the following axiom to the operational semantics:}\\
                    \inferrule{}{}
                    {\step{if t$_1$ then t$_2$ else t$_3$}{t$_2$}}
                    \begin{enumerate}
                        \item Progress: It still valid
                        \item Preserve: It still valid
                    \end{enumerate}
                \item[(d)]{Add the following rules:}
                    \inferrule{}{}
                    {\step{false + false }{false}}
                    \inferrule{}{}
                    {\step{true + true}{true}}
                    \inferrule{}{$t1:Bool$ \and $t2:Bool$}
                    {{t$_1$ + t$_2$:$Int$}}
                    \begin{enumerate}
                        \item Progress: It is still valid
                        \item Preserve: Consider the example: \verb|true + true = true|\\ But true is not a type of Int
                    \end{enumerate}
                \item[(e)]{Add the following rules:}
                    \inferrule{}{}
                    {\step{if 0 then t$_2$ else t$_3$ }{t$_2$}}
                    \inferrule{}{\tt t$_1$:Int \and t$_2$:T \and t$_3$:T}
                    {\tt if t$_1$ then t$_2$ else t$_3$:T}

                \begin{enumerate}
                        \item Progress: Consider the example: \verb|if 20 then true else false |\\ Then, t1 = 1 has type of Int.
                        It, however, does not step.
                        \item Preserve: It is still valid since for the first inference rule in this example does not take a step
                    \end{enumerate}

            \end{enumerate}




%
%
%        \item[3] {For every term $t$, either $t$ is a value or there exists a term $t$ such that $t$ $\rightarrow$ $t'$}\\
%                \textit{Induction hypothesis:} For every $t_0$, where $t_0$ is a subterm of t, then  either \verb|t|$_0$ is a value or there exists a term \verb|t|$_0'$  such that \verb|t|$_0$  $\rightarrow$ \verb|t|$_0'$ .
%                \begin{proof}
%                    Case analysis on form of t
%                    \begin{itemize}
%                        \item Case: $t$ = True , then $t$ is a value
%                        \item Case: $t$ = False, then $t$ is a value
%                        \item Case: $t$ = if $t_1$ then $t_2$ else $t_3$ :\\
%                                By IH, $t_1$ is either value,which are true or false, or there exists a term $t_1'$  such that $t_1$  $\rightarrow$ $t_1'$ \\
%                                $t$ = if true then $t_2$ else $t_3$ then $t'$ = $t2$ \\
%                                $t$ = if false then $t_2$ else $t_3$ then $t'$ = $t3$ \\
%                                $t$ = if $t_1$ then $t_2$ else $t_3$ then $t'$ = $t1'$ \\
%
%                    \end{itemize}
%                \end{proof}
%        \item[4] {Suppose we want to change the evaluation strategy of our language of booleans above so that the $then$ and $else$ branches of an $if$ expression are evaluated (in that order) before the guard is evaluated.Provide a new small-step operational semantics for the language that has this behavior.} \\
%            \inferax{E-IfTrue}{\step{if true then t$_2$ else t$_3$}{t$_2$}}
%            \inferax{E-IfFalse}{\step{if false then t$_2$ else t$_3$}{t$_3$}}
%            \inferrule{E-IfThen}{\step{t$_2$}{t$_2'$}}
%            {\step{if t$_1$ then t$_2$ else t$_3$}{if t$_1$ then t$_2'$ else t$_3$}}
%            \inferrule{E-IfElse}{\step{t$_3$}{t$_3'$}}
%            {\step{if t$_1$ then t$_2$ else t$_3$}{if t$_1$ then t$_2$ else t$_3'$}}
%            \inferrule{E-IfThen}{\step{t$_1$}{t$_1'$}}
%            {\step{if t$_1$ then v$_2$ else v$_3$}{if t$_1'$ then v$_2$ else v$_3$}}
%        \item[5] {Sol}\\
%            By induction on the derivation of t $\rightarrow$ t'\\
%            \textit{Induction hypothesis:} If $t_0$ $\rightarrow$ $t_0'$ and $t_0$ $\rightarrow$ $t_0''$, where $t_0$ is a sub-derivation of $t$ $\rightarrow$ $t'$, then $t_0'=t_0''$
%            \begin{proof}
%                Case analysis on the root rule in the derivation of $t \rightarrow t'$.
%            \begin{itemize}
%                \item Case: E-IFTRUE , then t = if $t_1$ then $t_2$ else $t_3$, where $t_1$ = true.\\ The last rule in the derivation of $t$ $\rightarrow$ $t''$ can be the following cases
%                    \begin{itemize}
%                        \item E-IFFALSE: $t_1$ is false: Contradiction($t_1$ = true)
%                        \item E-IF: $t_1$ $\rightarrow$ $t_1'$ : Contradiction ( True $t_1$ doesn't step )
%                        \item E-IFTRUE: The last rule immediate follows that $t' = t''$
%                    \end{itemize}
%                \item Case: $t$ = False: Similarly, the last rule in the derivation of $t \rightarrow t''$ must be the same.
%                \item Case: $t$ = if $t_1$ then $t_2$ else $t_3$, then $t_1 \rightarrow t_1'$ for some $t_1'$ \\
%                    By the same reasoning above, the last rule of $t \rightarrow t''$ can only be E-IF and $t1 \rightarrow t1'$ for some $t1''$
%                \\ By IH on $t_1$, $t_1' = t_1''$,\\  $t$ = if $t_1'$ then $t_2$ else $t_3$ = $t''$ = if $t_1''$ then $t_2$ else $t_3$
%
%            \end{itemize}
%        \end{proof}
%        \item[6] {Sol}\\
%            \begin{enumerate}
%                \item[(a)] \\
%                 if $true + 1$ then $true$ else $false$
%                \item[(b)] \\
%                if $true$ then $true>1$ else $false>1$
%            \end{enumerate}
%        \item[7] {Sol} \\
%                \begin{enumerate}
%                     \item[(a)] \\
%                     No
%                     \item[(b)] \\
%                     if $true$ then $2>1$ else $false>1$
%                \end{enumerate}


    \end{description}

%    \section{Language of Booleans and Integers}
%
%    \subsection{Syntax}
%
%    \begin{tt}
%        \begin{tabular}{rrl}
%            t & ::= & b $\mid$ if t then t else t \\
%            &$\mid$& n $\mid$ t + t $\mid$ t > t \\
%            b & ::= & true $\mid$ false \\
%            n & ::= & \mbox{\rm integer constant} \\
%            v & ::= & b $\mid$ n \\
%        \end{tabular}
%    \end{tt}
%
%
%    \subsection{Small-Step Operational Semantics}
%
%
%    \inferax{E-IfTrue}{\step{if true then t$_2$ else t$_3$}{t$_2$}}
%
%    \inferax{E-IfFalse}{\step{if false then t$_2$ else t$_3$}{t$_3$}}
%
%    \inferrule{E-If}{\step{t$_1$}{t$_1'$}}
%    {\step{if t$_1$ then t$_2$ else t$_3$}{if t$_1'$ then t$_2$ else t$_3$}}
%
%
%    \inferrule{E-Plus1}{\step{t$_1$}{t$_1'$}}
%    {\step{t$_1$ + t$_2$}{t$_1'$ + t$_2$}}
%
%
%    \inferrule{E-Plus2}{\step{t$_2$}{t$_2'$}}
%    {\step{v$_1$ + t$_2$}{v$_1$ + t$_2'$}}
%
%
%    \inferrule{E-PlusRed}{n = n$_1$ $[\![+]\!]$ n$_2$}
%    {\step{n$_1$ + n$_2$}{n}}
%
%    \inferrule{E-GT1}{\step{t$_1$}{t$_1'$}}
%    {\step{t$_1$ > t$_2$}{t$_1'$ > t$_2$}}
%
%
%    \inferrule{E-GT2}{\step{t$_2$}{t$_2'$}}
%    {\step{v$_1$ > t$_2$}{v$_1$ > t$_2'$}}
%
%
%    \inferrule{E-GTRed}{b = n$_1$ $[\![>]\!]$ n$_2$}
%    {\step{n$_1$ > n$_2$}{b}}


\end{document}
