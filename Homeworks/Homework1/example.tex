\documentclass{article}
\topmargin = -0.5in
\oddsidemargin = 0in
\evensidemargin = \oddsidemargin
\textwidth = 6.5in
\textheight = 9in
\usepackage{times}
\usepackage{bcprules}
\usepackage{amsthm}
\usepackage{amsmath}

\newcommand{\step}[2]{{\tt #1} $\longrightarrow$ {\tt #2}}

\newcommand{\inferrule}[3]{\infrule[#1]{\mbox{#2}}{\mbox{#3}}}
\newcommand{\inferax}[2]{\infrule[#1]{\mbox{}}{\mbox{#2}}}

\newtheorem{example}{Example}

\title{Homework 1}

\author{Yan-Ru Jhou}

\date{Jan 14th, 2020}

\begin{document}

    \maketitle

    \begin{description}
        \item[2]{ \verb|if (1+2) > 3 then 4+5 else 6+7|}
            \begin{enumerate}
                \item[(a)] {Provide a derivation tree, using our rules above, for one step of execution of the program P .}
                    \\

                    \inferrule{E-If}{\inferrule{E-PlusRed}{3 = 1 $[\![+]\!]$ 2}
                    {\step{($(1+2)>3$}{$3>3$}}}
                    {\step{if $(1+2) > 3$ then $4+5$ else $6+7$}{if $3>3$ then $4+5$ else $6+7$}}

                \item[(b)]{List the sequence of terms that P steps to according to our rules, up to and including the final normal form.
                You do not need to provide the derivation trees for each step.
                }
                    \\
                    \verb|if (1+2) > 3 then 4+5 else 6+7| \\
                    \verb|if 3 > 3 then 4+5 else 6+7| \\
                    \verb|if false then 4+5 else 6+7| \\
                    \verb|6 + 7| \\
                    \verb|13| \\
            \end{enumerate}
        \item[2] {For every term $t$, either $t$ is a value or there exists a term $t$ such that $t$ $\rightarrow$ $t'$}\\
                \textit{Induction hypothesis:} For every $t_0$, where $t_0$ is a subterm of t, then  either \verb|t|$_0$ is a value or there exists a term \verb|t|$_0'$  such that \verb|t|$_0$  $\rightarrow$ \verb|t|$_0'$ .
                \begin{proof}
                    Case analysis on form of t
                    \begin{itemize}
                        \item Case: $t$ = True , then $t$ is a value
                        \item Case: $t$ = False, then $t$ is a value
                        \item Case: $t$ = if $t_1$ then $t_2$ else $t_3$ :\\
                                By IH, $t_1$ is either value,which are true or false, or there exists a term $t_1'$  such that $t_1$  $\rightarrow$ $t_1'$ \\
                                $t$ = if true then $t_2$ else t$t_3$ then $t'$ = $t2$ \\
                                $t$ = if false then $t_2$ else t$t_3$ then $t'$ = $t3$ \\
                                $t$ = if $t_1$ then $t_2$ else t$t_3$ then $t'$ = $t1'$ \\

                    \end{itemize}
                \end{proof}
        \item[3] {Suppose we want to change the evaluation strategy of our language of booleans above so that the then and else branches of an if expression are evaluated (in that order) before the guard is evaluated.Provide a new small-step operational semantics for the language that has this behavior.} \\
            \inferax{E-IfTrue}{\step{if true then t$_2$ else t$_3$}{t$_2$}}
            \inferax{E-IfFalse}{\step{if false then t$_2$ else t$_3$}{t$_3$}}
            \inferrule{E-IfThen}{\step{t$_2$}{t$_2'$}}
            {\step{if t$_1$ then t$_2$ else t$_3$}{if t$_1$ then t$_2'$ else t$_3$}}
            \inferrule{E-IfElse}{\step{t$_3$}{t$_3'$}}
            {\step{if t$_1$ then t$_2$ else t$_3$}{if t$_1$ then t$_2$ else t$_3'$}}
            \inferrule{E-IfThen}{\step{t$_1$}{t$_1'$}}
            {\step{if t$_1$ then v$_2$ else v$_3$}{if t$_1'$ then v$_2$ else v$_3$}}

    \end{description}

    \section{Language of Booleans and Integers}

    \subsection{Syntax}

    \begin{tt}
        \begin{tabular}{rrl}
            t & ::= & b $\mid$ if t then t else t \\
            &$\mid$& n $\mid$ t + t $\mid$ t > t \\
            b & ::= & true $\mid$ false \\
            n & ::= & \mbox{\rm integer constant} \\
            v & ::= & b $\mid$ n \\
        \end{tabular}
    \end{tt}


    \subsection{Small-Step Operational Semantics}


    \inferax{E-IfTrue}{\step{if true then t$_2$ else t$_3$}{t$_2$}}

    \inferax{E-IfFalse}{\step{if false then t$_2$ else t$_3$}{t$_3$}}

    \inferrule{E-If}{\step{t$_1$}{t$_1'$}}
    {\step{if t$_1$ then t$_2$ else t$_3$}{if t$_1'$ then t$_2$ else t$_3$}}


    \inferrule{E-Plus1}{\step{t$_1$}{t$_1'$}}
    {\step{t$_1$ + t$_2$}{t$_1'$ + t$_2$}}


    \inferrule{E-Plus2}{\step{t$_2$}{t$_2'$}}
    {\step{v$_1$ + t$_2$}{v$_1$ + t$_2'$}}


    \inferrule{E-PlusRed}{n = n$_1$ $[\![+]\!]$ n$_2$}
    {\step{n$_1$ + n$_2$}{n}}

    \inferrule{E-GT1}{\step{t$_1$}{t$_1'$}}
    {\step{t$_1$ > t$_2$}{t$_1'$ > t$_2$}}


    \inferrule{E-GT2}{\step{t$_2$}{t$_2'$}}
    {\step{v$_1$ > t$_2$}{v$_1$ > t$_2'$}}


    \inferrule{E-GTRed}{b = n$_1$ $[\![>]\!]$ n$_2$}
    {\step{n$_1$ > n$_2$}{b}}


\end{document}
